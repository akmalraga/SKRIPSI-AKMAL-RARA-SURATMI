% packages.tex
\usepackage[utf8]{inputenc}
\usepackage{amsmath, amssymb}
\usepackage{graphicx}
\usepackage{eso-pic}
\usepackage{hyperref}
\usepackage{url}
\usepackage{pgfplots}
\pgfplotsset{compat=newest}
\usepackage{geometry}
\usepackage{setspace}
\usepackage{caption}
\usepackage{helvet}
\renewcommand{\familydefault}{\sfdefault}

\usepackage{fancyhdr}

% Header kanan atas
\pagestyle{fancy}
\fancyhf{}
\fancyhead[R]{\thepage}
\renewcommand{\headrulewidth}{0pt}

% Bab dengan angka Romawi
\renewcommand{\thechapter}{\Roman{chapter}}

\usepackage{subcaption}
\usepackage[utf8]{inputenc}
\usepackage{titlesec}
\titlespacing{\chapter}{5pt}{2pt}{2pt}
\usepackage[indonesia]{babel}

\usepackage[backend=biber,style=authoryear,language=auto,autolang=other]{biblatex}
\addbibresource{ref.bib}

\renewcommand{\contentsname}{DAFTAR ISI}
\renewcommand{\listfigurename}{DAFTAR GAMBAR}
\renewcommand{\listtablename}{DAFTAR TABEL}
\renewcommand{\bibname}{DAFTAR PUSTAKA}   
\renewcommand{\chaptername}{BAB}

\titleformat{\chapter}[display]
  {\normalfont\bfseries\normalsize\centering\normalsize}
  {\chaptername\ \thechapter}
  {2pt}
  {\normalsize}

\titleformat{\section}
  [hang] % atau display, block, runin sesuai kebutuhan
  {\normalfont\bfseries\normalsize} % format font section (bold, normal size)
  {\thesection} % nomor section
  {1em} % jarak antara nomor dan judul
  {}

\titleformat{\subsection}
  [hang] % atau display, block, runin sesuai kebutuhan
  {\normalfont\bfseries\normalsize} % format font section (bold, normal size)
  {\thesubsection} % nomor section
  {1em} % jarak antara nomor dan judul
  {}

\usepackage{listings}
\usepackage[table,dvipsnames]{xcolor}
\lstset{
  basicstyle=\ttfamily\small,
  numbers=left,
  numberstyle=\tiny,
  numbersep=5pt,
  backgroundcolor=\color{gray!10},
  keywordstyle=\color{blue},
  commentstyle=\color{green!50!black},
  stringstyle=\color{red},
  breaklines=true,
  frame=single,
}

\usepackage{siunitx}
\usepackage{float}
\usepackage{booktabs}
\usepackage{lipsum} % for dummy text
\usepackage{ragged2e}
%\usepackage[english, indonesia]{babel}
\usepackage{tikz}
\usetikzlibrary{arrows.meta, positioning, shapes.geometric, calc}
\usepackage{tabularx} 
