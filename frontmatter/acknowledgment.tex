\chapter{Derivasi Matematis}
\section{Tight Binding Model}
\label{sec:tbmodel}
Persamaan Schrodinger untuk potensial \textit{single atom}:$U(r)$
\begin{align}
    \left[\frac{p^2}{2m} + U(r) \right]\phi(r)=E_0 \phi(r)
    \\ \phi^*(r)\left[\frac{p^2}{2m} + U(r) \right]\phi(r)=\phi^*(r)E_0\phi(r)
\end{align}
Dengan menambahkan pengaruh dari atom lain:
\begin{align}
    \psi_k(r)=\sum_j b_k(r_j) \phi_k(r-r_j) \\
    \psi(r+R_l)=e^{ik.R_l}\psi(r), && (Teori && Bloch)\\
    \sum_j b_k(r_j) \phi(r-r_j+R_l)=e^{ik.R_l}\sum_j b_k(r_j) \phi(r-r_j)
\end{align}
Dengan mengsubtitusi $R_p=r_j-R_l$
\begin{align}
    b_k(R_p+R_l)=e^{ik.R_l}b_k(R_p)
\\  b_k(R_l)=e^{ik.R_l}b_k(0)\\
 b_k(r_j)=e^{ik.r_j}b_k(0)
\end{align}
Normalisasi Fungsi Gelombang:
\begin{align}
    \int{d^3r\psi^*_k(\textbf{r})\psi_k(\textbf{r})}=1\\
    1=\sum_{r_j}b_k^*(r_j) \sum_{R_l}b_k^*(R_l)\int d^3r\phi^*_k(r-r_j)\phi_k(r-R_l)\\
    1=b_k^*(0)b_k(0)\sum_{r_j}e^{ik.r_j}\sum_{R_l}e^{ik.R_l}\int d^3r\phi^*_k(r-r_j)\phi_k(r-R_l)\\
    1=Nb_k^*(0)b_k(0)\sum_{R_p}e^{-ik.R_p}\int d^3r\phi^*_k(r-R_p)\phi_k(r)\\
    \left[1=Nb_k^*(0)b_k(0)\sum_{R_p}e^{-ik.R_p}\int d^3r\phi^*_k(r-R_p)\phi_k(r)\right]^*\\
    1=Nb_k^*(0)b_k(0)\sum_{R_p}e^{ik.R_p}\int d^3r\phi^*_k(r)\phi_k(r-R_p)
    \\
    b_k^*(0)b_k(0)=\frac{1}{N}.\frac{1}{1+\sum_{R_p}e^{ik.R_p}\alpha_k(R_p)}
    \\ b_k(0) \approx \frac{1}{\sqrt{N}}
    \\ \psi_k(r)=\frac{1}{\sqrt{N}}\sum_j e^{ik.r_j} \phi_k(r-r_j)
\end{align}
Maka \textit{first order correction} dari energi:
\begin{align}
    \psi_k^*(r)H\psi_k=\frac{1}{N}\sum_j \sum_m e^{ik.(r_j-r_m)}\int\phi^*(r-r_m)H\phi(r-r_j)dr
    \\ \psi_k^*(r)H\psi_k= \sum_n e^{-ik.\rho(n)}\int\phi^*(r)H\phi(r - r_j)
\end{align}
Misal $\gamma = \phi^*(r)H\phi(r - r_j) $, dan basis orbital lokal $|f_n \rangle = | \phi_a (r - r_j) \rangle$ maka Hamiltonian Tight Binding dapat kita modelkan sebagai
\begin{align}
  H = E_0 \sum_n |f_n \rangle \langle f_n | + \gamma \sum_n [|f_n \rangle \langle f_{n+1} | + |f_{n+1}\rangle \langle f_n | ] \\ 
  H = \sum_n a_n |f_n \rangle \langle f_n | +  \sum_n b_{n+1} [|f_n \rangle \langle f_{n+1} | + |f_{n+1}\rangle \langle f_n | ] \\ 
\end{align}

\section{Operator Kreasi dan Anihilasi}
Hamiltonian osilator harmonik kuantum satu dimensi:
\begin{align}
H = \frac{p^2}{2m} + \frac{1}{2}m\omega^2 x^2
\end{align}
Faktorisasi persamaan Hamiltonian dengan mendefinisikan operator a dan a†:
\begin{align}
a &= \sqrt{\frac{m\omega}{2\hbar}} \left( x + \frac{i}{m\omega} p \right) \
a^\dagger &= \sqrt{\frac{m\omega}{2\hbar}} \left( x - \frac{i}{m\omega} p \right)
\end{align}
Komutator dari operator a dan a† dengan relasi [x,p]=iℏ:
\begin{align}
[a, a^\dagger] &= \frac{m\omega}{2\hbar} \left[ x + \frac{i}{m\omega} p, x - \frac{i}{m\omega} p \right] \\
[a, a^\dagger] &= \frac{m\omega}{2\hbar} \left( -\frac{i}{m\omega}[x,p] + \frac{i}{m\omega}[p,x] \right) \\
[a, a^\dagger] &= \frac{m\omega}{2\hbar} \left( -\frac{i}{m\omega}(i\hbar) + \frac{i}{m\omega}(-i\hbar) \right) \\
[a, a^\dagger] &= \frac{m\omega}{2\hbar} \left( \frac{\hbar}{m\omega} + \frac{\hbar}{m\omega} \right) = 1
\end{align}
Substitusi perkalian operator ke dalam Hamiltonian:
\begin{align}
a^\dagger a &= \frac{m\omega}{2\hbar} \left( x - \frac{i}{m\omega} p \right) \left( x + \frac{i}{m\omega} p \right) \\
a^\dagger a &= \frac{m\omega}{2\hbar} \left( x^2 + \frac{p^2}{m^2\omega^2} + \frac{i}{m\omega}[x,p] \right) \\
a^\dagger a &= \frac{m\omega}{2\hbar} x^2 + \frac{p^2}{2m\hbar\omega} - \frac{1}{2} \\
\hbar\omega \left( a^\dagger a + \frac{1}{2} \right) &= \frac{p^2}{2m} + \frac{1}{2}m\omega^2 x^2
\end{align}
Hamiltonian dalam bentuk operator jumlah partikel N=a†a:
\begin{align}
H = \hbar\omega \left( a^\dagger a + \frac{1}{2} \right)
\end{align}
Komutator Hamiltonian terhadap operator a dan a†:
\begin{align}
[H, a] &= \hbar\omega [a^\dagger a, a] = -\hbar\omega a \
[H, a^\dagger] &= \hbar\omega [a^\dagger a, a^\dagger] = \hbar\omega a^\dagger
\end{align}

Aplikasi operator pada nilai eigen energi H∣n⟩=En​∣n⟩:
\begin{align}
H (a^\dagger |n\rangle) &= (E_n + \hbar\omega) (a^\dagger |n\rangle) \
H (a |n\rangle) &= (E_n - \hbar\omega) (a |n\rangle)
\end{align}
Aksi operator kreasi (menciptakan partikel/eksitasi) dan anihilasi (memusnahkan):
\begin{align}
a^\dagger |n\rangle &= \sqrt{n+1} |n+1\rangle \
a |n\rangle &= \sqrt{n} |n-1\rangle
\end{align}
Dari operator ini kita dapat menuliskan kembali model Tight-Binding \ref{sec:tbmodel},
\[
\begin{aligned}
  H =\sum_i \epsilon a^{\dagger}_{i} a_{i} -t \sum_{\langle i,j \rangle} \left( a^{\dagger}_{i} a_{j} + \text{H.c.} \right) 
    \end{aligned}
\]

\section{Derivasi Fase Berry, Koneksi Berry, dan Kelengkungan Berry}

Misalkan Hamiltonian $H(R)$ berevolusi secara adiabatik, maka;
\begin{equation}
  H(\bar{R}(t)) = H(r_1(t), r_2(t), ...)
\end{equation}
Persamaan Schrodinger bergantung waktu;
\begin{align}
  H|\psi(t)\rangle = i \hbar \frac{\partial}{\partial t}|\psi(t)\rangle  \\
  |\psi(t)\rangle = \sum_n e^{-iE_nt/\hbar} |n\rangle \qquad ; \qquad H|n \rangle = E_n|n\rangle \\ 
  |\psi(t)\rangle = e^{-iE_nt/\hbar} |n \rangle
\end{align}
Kembali kepada hamiltonian $H(R)$
\begin{align}
  H(\bar{R}(t))|\psi(t)\rangle = i\hbar \frac{\partial}{\partial t} |\psi(t)\rangle  \\ 
  H(\bar{R}(t))|n, \bar{R}(t)\rangle = E_n(\bar{R}(t)) |n, \bar{R}(t)\rangle \\ 
    t = 0 \rightarrow |\psi(0) \rangle = |n, \bar{R}(0) \rangle
\end{align}
Apabila kita asumsikan bahwa solusi dari sistem ini hanya terdiri dari fase dinamis, maka:
\begin{equation}
  |\psi(t)\rangle = \exp \left[ -\frac{i}{\hbar} \int_0^t E_n (\bar{R}(s))ds \right] |n, \bar{R}(t) \rangle 
\end{equation}
subtitusi,
\begin{align}
  i\hbar \frac{\partial}{\partial t}|\psi(t)\rangle = E_n \left( \vec{R}(t) \right) \exp \left[ -\frac{i}{\hbar} \int_{0}^{t} E_n (\vec{R}(s)) ds \right] |n, \vec{R}(t) \rangle \\ 
  + \exp \left[ -\frac{i}{\hbar} \int_{0}^{t} E_n (\vec{R}(s)) ds \right] i\hbar \frac{\partial}{\partial t} |n, \vec{R}(t) \rangle \\ 
  = E_n \left( \vec{R}(t) \right) \exp \left[ -\frac{i}{\hbar} \int_{0}^{t} E_n (\vec{R}(s)) ds \right] |n, \vec{R}(t) \rangle \\ 
   + \exp \left[ -\frac{i}{\hbar} \int_{0}^{t} E_n (\vec{R}(s)) ds \right] i\hbar \dot{\vec{R}}(t) \frac{\partial}{\partial \vec{R}} |n, \vec{R}(t) \rangle \\ 
   i\hbar \frac{\partial}{\partial t}|\psi(t)\rangle = H(\vec{R}(t)) |\psi(t) \rangle \\ 
  + \exp \left[ -\frac{i}{\hbar} \int_{0}^{t} E_n (\vec{R}(s)) ds \right] i\hbar \dot{\vec{R}}(t) \frac{\partial}{\partial \vec{R}} |n, \vec{R}(t) \rangle  
\end{align}
$ i\hbar \dot{\vec{R}}(t) \frac{\partial}{\partial \vec{R}} |n, \vec{R}(t) \rangle $ harus nol untuk memenuhi persamaan Schrodinger. Untuk itu kita memerlukan tambahan fase pada solusi tebakan;
\begin{align}
  |\psi(t)\rangle = \exp[i\gamma_n(t)] \exp \left[ -\frac{i}{\hbar} \int_0^t E_n (\bar{R}(s))ds \right] |n, \bar{R}(t) \rangle  \\ 
i\hbar \frac{\partial}{\partial t} |\Psi(t) \rangle = -\hbar \frac{d\gamma_n(t)}{dt} |\Psi(t) \rangle + E_n \left( \vec{R}(t) \right) |\Psi(t) \rangle \\
+ \exp \left[ i\gamma_n(t) - \frac{i}{\hbar} \int_{0}^{t} E_n (\vec{R}(s)) ds \right] i\hbar \frac{\partial}{\partial t} |n, \vec{R}(t) \rangle \\ 
\end{align}  
Untuk memenuhi persamaan Schrodinger,$ -\hbar \frac{d\gamma_n(t)}{dt} |\Psi(t) \rangle$ dan $ \exp \left[ i\gamma_n(t) - \frac{i}{\hbar} \int_{0}^{t} E_n (\vec{R}(s)) ds \right] i\hbar \frac{\partial}{\partial t} |n, \vec{R}(t) \rangle $ harus menjadi nol.


\begin{align}
    -\hbar \frac{d\gamma_n(t)}{dt} |\Psi(t) \rangle &= \exp \left[ i\gamma_n(t) - \frac{i}{\hbar} \int_{0}^{t} E_n (\vec{R}(s)) ds \right] i\hbar \frac{\partial}{\partial t} |n, \vec{R}(t) \rangle \\ 
    -\hbar \frac{d\gamma_n(t)}{dt} |\Psi(t) \rangle &= \exp \left[ i\gamma_n(t) - \frac{i}{\hbar} \int_{0}^{t} E_n (\vec{R}(s)) ds \right] i\hbar \dot{\vec{R}} \cdot \nabla_{\vec{R}} |n, \vec{R}(t) \rangle \\ 
    \frac{d\gamma_n(t)}{dt} |n, \vec{R}(t) \rangle &= i \dot{\vec{R}} \cdot \nabla_{\vec{R}} |n, \vec{R}(t) \rangle \\ 
    \frac{d\gamma_n(t)}{dt} \langle n, \vec{R}(t)|n, \vec{R}(t) \rangle &= i \dot{\vec{R}} \cdot \langle n, \vec{R}(t)| \nabla_{\vec{R}} |n, \vec{R}(t) \rangle \\
    \frac{d\gamma_n(t)}{dt} &= i \dot{\vec{R}} \cdot \langle n, \vec{R}(t)| \nabla_{\vec{R}} |n, \vec{R}(t) \rangle \\ 
    \gamma_n(t) &= i \int_0^t ds \quad \dot{\vec{R}}(s)\langle n, \vec{R}(s)| \nabla_{\vec{R}} |n, \vec{R}(s) \rangle \\ 
                &= i \int_{\vec{R_1}}^{\vec{R_2}} d\vec{R} \; \langle n, \vec{R}| \nabla_{\vec{R}} |n, \vec{R} \rangle
  \end{align}
Fase Berry $\gamma_n$ harus real untuk memenuhi konversasi probabilitas gelombang.

\begin{align}
\gamma_n &= \int_{\vec{R}_1}^{\vec{R}_2} d\vec{R} \cdot i \langle n, \vec{R} | \nabla_{\vec{R}} | n, \vec{R} \rangle \\
&= \int_{s_1}^{s_2} ds \frac{d\vec{R}(s)}{ds} \cdot i \langle n, \vec{R} | \nabla_{\vec{R}} | n, \vec{R} \rangle \\
&\text{Fakta bahwa: } \frac{d}{ds} | n, \vec{R}(s) \rangle = \left( \nabla_{\vec{R}} | n, \vec{R} \rangle \right) \cdot \frac{d\vec{R}(s)}{ds} \\
\gamma_n &= \int_{s_1}^{s_2} ds \, i \left\langle n, \vec{R} \right| \frac{d}{ds} \left| n, \vec{R} \right\rangle
\end{align}

\fbox{$\langle n, \vec{R} | n, \vec{R} \rangle = 1$} \quad 
\fbox{$\frac{d}{ds} \langle n, \vec{R} | n, \vec{R} \rangle = 0$} \quad 
\fbox{$\frac{d}{ds} \langle n, \vec{R} | = \left( \frac{d}{ds} | n, \vec{R} \rangle \right)^\dagger$}

\begin{align*}
\frac{d}{ds} \langle n, \vec{R}(s) | n, \vec{R}(s) \rangle &= \frac{d}{ds} \langle n, \vec{R} | | n, \vec{R} \rangle + \langle n, \vec{R} | \frac{d}{ds} | n, \vec{R} \rangle \\
&= 2\text{Re} \left[ \langle n, \vec{R} | \frac{d}{ds} | n, \vec{R} \rangle \right] \\ 
&= 0
\end{align*}
Karena $\langle n, \vec{R} | \frac{d}{ds} | n, \vec{R} \rangle$ sama dengan 0, maka fase berry $\gamma_n$ harus real
\par 
Isi integral dari fase berry disebut koneksi Berry $\vec{A}_n(\vec{R})$ yang dapat dipandang sebagai sebuah medan vector.
\begin{equation}
  A_{n, \mu}(\vec{R}) = i \langle n, \vec{R} | \frac{\partial}{\partial r_\mu} | n, \vec{R} \rangle
\end{equation}

Misal koneksi Berry diberikan sebagai $\vec{A_n}(k) = i \langle n_{\vec{k}} | \nabla | n_{\vec{k}} \rangle$ dan mengaplikasikan projector $\hat{P}_{\vec{k}} = | n_{\vec{k}} \rangle \langle n_{\vec{k}} |$

\fbox{$\vec{A}_n(\vec{k}) = i \text{Tr} [\hat{P}_{\vec{k}} \nabla_{\vec{k}} \hat{P}_{\vec{k}}]$} \quad 
\fbox{$\hat{P}_{\vec{k}} = | n_{\vec{k}} \rangle \langle n_{\vec{k}} |$}

\begin{align*}
\gamma_n &= \int_{\vec{k}_1}^{\vec{k}_2} \vec{A}_n(\vec{k}) \cdot d\vec{k} \approx \sum_{j=0}^{N-1} \vec{A}_n(\vec{k}_j) \cdot d\vec{k}_j = i \sum_{j=0}^{N-1} \text{Tr} [\hat{P}_{\vec{k}_j} \nabla_{\vec{k}} \hat{P}_{\vec{k}_j}] \cdot d\vec{k}_j \\
&= i \sum_{j=0}^{N-1} \text{Tr} [\hat{P}_{\vec{k}_j} (\nabla_{\vec{k}} \hat{P}_{\vec{k}_j}) \cdot d\vec{k}_j] \approx i \sum_{j=0}^{N-1} \text{Tr} [\hat{P}_{\vec{k}_j} (\hat{P}_{\vec{k}_{j+1}} - \hat{P}_{\vec{k}_j})] \\
&= i \sum_{j=0}^{N-1} \langle n_{\vec{k}_j} | n_{\vec{k}_{j+1}} \rangle - 1
\end{align*}

\begin{center}
    \textcolor{red}{\textbf{Parallel transport}} \\
    \colorbox{yellow!20}{
        $\displaystyle \langle n_{\vec{k}_j} | n_{\vec{k}_{j+1}} \rangle = 1 - i \vec{A}_n(\vec{k}_j) \cdot d\vec{k}_j$
    }
\end{center}
\par 
Koneksi Berry $A_n(k)$ bersifat gauge-dependent, dimana transormasinya menghasilkan suku gradient fase $\nabla_k \phi (k)$$\vec{A}_n(\vec{k}) \xrightarrow{\text{Gauge}} \vec{A}_n(\vec{k}) + \nabla_{\vec{k}} \phi(k)$. Untuk itu, kelengkungan Berry diberikan untuk memberikan parameter yang gauge-invariant.
\begin{align*}
\vec{\Omega}_n(\vec{k}) &\equiv \nabla \times \vec{A}_n(\vec{k}) \\
\vec{\Omega}_n(\vec{k}) &\xrightarrow{\text{Gauge}} \vec{\Omega}_n(\vec{k})
\end{align*}

\section{Formulasi Wannier Center}

\subsection*{Bloch State}

\begin{align}
\psi_{n\mathbf{k}}(\mathbf{r})
&=
e^{i\mathbf{k}\cdot\mathbf{r}}
u_{n\mathbf{k}}(\mathbf{r}) \\
u_{n\mathbf{k}}(\mathbf{r}+\mathbf{R})
&=
u_{n\mathbf{k}}(\mathbf{r})
\end{align}

Normalisasi:
\begin{align}
\langle \psi_{n\mathbf{k}'} | \psi_{n\mathbf{k}} \rangle
=
\frac{(2\pi)^3}{V}
\delta(\mathbf{k}-\mathbf{k}')
\end{align}

\subsection*{Definisi Wannier Function}

\begin{align}
| w_{n0} \rangle
=
\frac{V}{(2\pi)^3}
\int_{\text{BZ}} d^3k \,
| \psi_{n\mathbf{k}} \rangle
\end{align}

Substitusi bentuk Bloch:

\begin{align}
| w_{n0} \rangle
=
\frac{V}{(2\pi)^3}
\int_{\text{BZ}} d^3k \,
e^{i\mathbf{k}\cdot\mathbf{r}}
| u_{n\mathbf{k}} \rangle
\end{align}

\subsection*{Identitas Operator Posisi}

\begin{align}
\mathbf{r} e^{i\mathbf{k}\cdot\mathbf{r}}
=
i \nabla_{\mathbf{k}} e^{i\mathbf{k}\cdot\mathbf{r}}
\end{align}

\subsection*{Aksi Operator Posisi}

\begin{align}
\mathbf{r}
| w_{n0} \rangle
&=
\frac{V}{(2\pi)^3}
\int_{\text{BZ}} d^3k \,
\mathbf{r}
e^{i\mathbf{k}\cdot\mathbf{r}}
| u_{n\mathbf{k}} \rangle \\
&=
i
\frac{V}{(2\pi)^3}
\int_{\text{BZ}} d^3k \,
\nabla_{\mathbf{k}}
\left(
e^{i\mathbf{k}\cdot\mathbf{r}}
\right)
| u_{n\mathbf{k}} \rangle
\end{align}

Integrasi parsial di ruang $\mathbf{k}$:

\begin{align}
&=
i
\frac{V}{(2\pi)^3}
\int_{\text{BZ}} d^3k \,
\nabla_{\mathbf{k}}
\left(
e^{i\mathbf{k}\cdot\mathbf{r}}
| u_{n\mathbf{k}} \rangle
\right) \\
&\quad
-
i
\frac{V}{(2\pi)^3}
\int_{\text{BZ}} d^3k \,
e^{i\mathbf{k}\cdot\mathbf{r}}
\nabla_{\mathbf{k}}
| u_{n\mathbf{k}} \rangle
\end{align}

Dengan periodisitas pada batas BZ, suku permukaan hilang:

\begin{align}
\mathbf{r}
| w_{n0} \rangle
=
i
\frac{V}{(2\pi)^3}
\int_{\text{BZ}} d^3k \,
e^{i\mathbf{k}\cdot\mathbf{r}}
\nabla_{\mathbf{k}}
| u_{n\mathbf{k}} \rangle
\end{align}

\subsection*{Ekspektasi Nilai Posisi}

\begin{align}
\langle w_{n0} |
\mathbf{r}
| w_{n0} \rangle
&=
i
\left(
\frac{V}{(2\pi)^3}
\right)^2
\int_{\text{BZ}} d^3k
\int_{\text{BZ}} d^3k'
\,
\langle u_{n\mathbf{k}'} |
e^{-i\mathbf{k}'\cdot\mathbf{r}}
e^{i\mathbf{k}\cdot\mathbf{r}}
\nabla_{\mathbf{k}}
| u_{n\mathbf{k}} \rangle
\end{align}

Gunakan ortogonalitas Bloch state:

\begin{align}
\langle \psi_{n\mathbf{k}'} | \psi_{n\mathbf{k}} \rangle
=
\frac{(2\pi)^3}{V}
\delta(\mathbf{k}-\mathbf{k}')
\end{align}

Sehingga diperoleh:

\begin{align}
\langle w_{n0} |
\mathbf{r}
| w_{n0} \rangle
=
i
\frac{V}{(2\pi)^3}
\int_{\text{BZ}} d^3k \,
\langle u_{n\mathbf{k}} |
\nabla_{\mathbf{k}}
u_{n\mathbf{k}} \rangle
\end{align}

\subsection*{Definisi Berry Connection}

\begin{align}
\mathbf{A}_n(\mathbf{k})
=
i
\langle u_{n\mathbf{k}} |
\nabla_{\mathbf{k}}
u_{n\mathbf{k}} \rangle
\end{align}

\subsection*{Hasil Akhir}

\begin{align}
\boxed{
\mathbf{r}_n
=
\langle w_{n0} |
\mathbf{r}
| w_{n0} \rangle
=
\frac{V}{(2\pi)^3}
\int_{\text{BZ}}
d^3k \,
\mathbf{A}_n(\mathbf{k})
}
\end{align}

\chapter{Lampiran Kode Program}
\label{app:kode-program}
\section{Kode Utama Simulasi(Model Kane-Mele 3D)}
Berikut adalah kode utama yang memodelkan sistem yang kita teliti dalam lingkungan PyhTB:

\begin{lstlisting}[language=Python, caption={Model Stacking Graphene}, label={lst:kode-simulasi}]
import numpy as np
import matplotlib.pyplot as plt
from pythtb import tb_model

# Konstanta dan parameter
delta = 0.7
t = -1.0
soc_list = np.array([-0.054, -0.24])
rashba = 0.05
width = 10
nkr = 101
W = 15*soc_list
n_avg = 10  # jumlah sample untuk averaging

sigma_z = np.array([0., 0., 0., 1.])
sigma_x = np.array([0., 1., 0., 0])
sigma_y = np.array([0., 0., 1., 0])
r3h = np.sqrt(3.0) / 2.0
sigma_a = 0.5 * sigma_x - r3h * sigma_y
sigma_b = 0.5 * sigma_x + r3h * sigma_y
sigma_c = -1.0 * sigma_x

def set_model(t, soc, rashba, delta, W):
    lat = [[1, 0, 0], [0.5, np.sqrt(3.0)/2.0, 0.0], [0.0, 0.0, 1.0]]
    orb = [[1./3., 1./3., 0.0], [2./3., 2./3., 0.0]]
    model = tb_model(3, 3, lat, orb, nspin=2)

    disorder_values = np.random.uniform(-W/2, W/2, size=len(orb))  
    onsite_energies = [delta + disorder_values[i] if i % 2 == 0 else -delta + disorder_values[i] for i in range(len(orb))]
    model.set_onsite(onsite_energies)

    for lvec in ([0, 0, 0], [-1, 0, 0], [0, -1, 0]):
        model.set_hop(t, 0, 1, lvec)
    for lvec in ([1, 0, 0], [-1, 1, 0], [0, -1, 0]):
        model.set_hop(soc * 1.j * sigma_z, 0, 0, lvec)
    for lvec in ([-1, 0, 0], [1, -1, 0], [0, 1, 0]):
        model.set_hop(soc * 1.j * sigma_z, 1, 1, lvec)
    model.set_hop(0.3 * soc * 1j * sigma_z, 1, 1, [0, 0, 1])
    model.set_hop(-0.3 * soc * 1j * sigma_z, 0, 0, [0, 0, 1])
    model.set_hop(1.j * rashba * sigma_a, 0, 1, [0, 0, 0], mode="add")
    model.set_hop(1.j * rashba * sigma_b, 0, 1, [-1, 0, 0], mode="add")
    model.set_hop(1.j * rashba * sigma_c, 0, 1, [0, -1, 0], mode="add")

    return model
\end{lstlisting}

\section{Kode Tambahan: Analisis Data}
\textit{Syntax} tambahan untuk menganalisis hasil simulasi:

\begin{lstlisting}[language=Python, caption={Kode Analisis DOS}, label={lst:kode-analisis DOS}]

dos_at_zero = []

for W in W_values:
    all_eigenvalues = []
    for _ in range(n_samples):
        # Bangun model dengan disorder W
        my_model = set_model(t, soc, rashba, delta, W)
        
        # Potong model menjadi ribbon
        ribbon_model = my_model.cut_piece(width, fin_dir=1, glue_edgs=False)
        
        # Hitung eigenenergi
        (k_vec, k_dist, k_node) = ribbon_model.k_path(
            [[0.,0.], [2./3.,1./3.], [.5,.5], [1./3.,2./3.], [0.,0.]],
            nkr, report=False
        )
        rib_eval = ribbon_model.solve_all(k_vec)
        all_eigenvalues.append(rib_eval.flatten())
    
    # Gabungkan semua eigenenergi
    combined_eval = np.concatenate(all_eigenvalues)
    
    # Hitung histogram dan ekstrak DOS pada E=0
    hist, bin_edges = np.histogram(combined_eval, bins=50, range=(-4., 4.), density=True)
    bin_centers = 0.5 * (bin_edges[1:] + bin_edges[:-1])
    idx_zero = np.argmin(np.abs(bin_centers))  # Indeks bin terdekat E=0
    dos_at_zero.append(hist[idx_zero])

# Menyimpan data ke dalam file CSV
with open('dos_at_zero.csv', mode='w', newline='') as file:
    writer = csv.writer(file)
    writer.writerow(["W/soc", "DOS at E=0"])  # Menulis header
    for W, dos in zip(W_values / soc, dos_at_zero):
        writer.writerow([W, dos])  # Menulis setiap baris data


# Plot hasil
#z2 =ribbon_model.z2_invariant()
#print(f"Z2 invariant: {z2}")
plt.plot(W_values/soc, dos_at_zero, 'o-', label='Simulasi')
plt.axvline(x=2.5, c='r', ls='--', label='Prediksi $W_c=2.5\lambda_{SO}$')
plt.xlabel("$W/\lambda_{SO}$")
plt.ylabel("DOS pada $E=0$")
plt.legend()
plt.show()


\end{lstlisting}

\begin{lstlisting}[language=Python, caption={Kode Analisis Pita Energi}, label={lst:kode-analisis Pita Energi}]

fig, ax = plt.subplots(1, 2, figsize=(10, 4))

for je, soc_val in enumerate(soc_list):
    eval_total = None
    for _ in range(n_avg):
        my_model = set_model(t, soc_val, rashba, delta, W)
        ribbon_model = my_model.cut_piece(width, fin_dir=1, glue_edgs=False)
        path = [[0.,0.],[2./3.,1./3.],[.5,.5],[1./3.,2./3.],[0.,0.]]
        (k_vec, k_dist, k_node) = ribbon_model.k_path(path, nkr, report=False)
        rib_eval = ribbon_model.solve_all(k_vec, eig_vectors=False)

        if eval_total is None:
            eval_total = np.array(rib_eval)
        else:
            eval_total += np.array(rib_eval)

    rib_eval_avg = eval_total / n_avg
    nbands = rib_eval_avg.shape[0]
    ax1 = ax[je]
    ax1.set_xlim([0, k_node[-1]])
    ax1.set_xticks(k_node)
    ax1.set_xticklabels([r'$\Gamma$', r'$K$', r'$M$', r"$K'$", r'$\Gamma$'])
    ax1.set_ylim(-5, 5)
    ax1.set_ylabel("Averaged Band Structure (010)")
    ax1.set_title(f"SOC = {soc_val:.3f}")
    
    for i in range(len(k_vec)):
        ax1.scatter([k_dist[i]] * nbands, rib_eval_avg[:, i], s=2, c='black', alpha=0.5)

plt.tight_layout()
plt.savefig("averaged_band_structure.pdf")
plt.show()


\end{lstlisting}


\begin{lstlisting}[language=Python, caption={Kode Diagram Fasa}, label={lst:Kode Diagram Fasa}]

import matplotlib.pyplot as plt
import numpy as np

# Define phase boundary values
Wc_onsite = 6.0  # threshold for onsite disorder
Wc_hopping = 5.0  # threshold for hopping disorder

# Create grid
x = np.linspace(1, 10, 100)  # Onsite disorder
y = np.linspace(1, 10, 100)  # Hopping disorder
X, Y = np.meshgrid(x, y)

# Define trivial phase condition: either x > Wc_onsite or y > Wc_hopping
Z = np.zeros_like(X)
Z[(X > Wc_onsite) | (Y > Wc_hopping)] = 1  # Anderson Insulators
Z[(X <= Wc_onsite) & (Y <= Wc_hopping)] = 0  # Topological phase

# Plot the phase diagram
fig, ax = plt.subplots(figsize=(6, 5))
c = ax.contourf(X, Y, Z, levels=[-0.1, 0.5, 1.1], colors=['red', 'black'], alpha=0.8)
ax.contour(X, Y, Z, levels=[0.5], colors='k', linewidths=1)

# Annotations and labels
ax.set_xlabel("Onsite Disorder W (x)")
ax.set_ylabel("Hopping Disorder W (y)")
ax.set_title("Phase Diagram: Topological vs Anderson Insulators")

# Color legend
from matplotlib.patches import Patch
legend_elements = [Patch(facecolor='red', label='Topological Phase'),
                   Patch(facecolor='black', label='Anderson Insulator Phase')]
ax.legend(handles=legend_elements, loc='upper left')

plt.tight_layout()
plt.show()


\end{lstlisting}

\lstset{
  basicstyle=\ttfamily\small,
  backgroundcolor=\color{gray!10},
  frame=single,
  caption={Data DOS at E=0 terhadap kekuatan disorder (W/soc)},
  label={lst:data-dos},
}

\begin{lstlisting}
W/soc,DOS at E=0
0.0,0.0
1.1111111111111112,0.0
2.2222222222222223,0.0
3.3333333333333335,0.0005538366336633659
4.444444444444445,0.0019863861386138597
5.555555555555555,0.00779702970297029
6.666666666666667,0.017558787128712856
7.777777777777779,0.023363242574257405
8.88888888888889,0.03208796943121178
10.0,0.03748905795087555
\end{lstlisting}


