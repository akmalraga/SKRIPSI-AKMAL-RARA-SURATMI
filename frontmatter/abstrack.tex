\newpage
\begin{abstract}
\justifying AKMAL SURATMI. \textbf{Pengaruh Disorder Anderson Terhadap Struktur Spektrum Energi \textit{Quantum Spin Hall} dalam Model Grafena Kane Mele 3D}.(dibimbing oleh Tasrief surungan) 
\par
\noindent
\textbf{LATAR BELAKANG}. Penelitian ini menganalisis perubahan spektrum energi \textbf{Quantum Spin Hall Insulators} (QSHI) pada model Kane-Mele grafena tiga dimensi (3D) yang disusun secara periodik sepanjang sumbu-$z$ di bawah pengaruh lokalisasi Anderson. \textbf{TUJUAN}. Tujuan utama penelitian ini adalah meninjau pengaruh kekuatan disorder ($W$) pada sistem 3D serta mengamati degradasi \textit{helical edge states} dalam struktur pita energi. \textbf{METODE}. Model 3D dikonstruksi dengan menumpuk lapisan-lapisan grafena dua dimensi (2D) menggunakan parameter hopping antarlapisan ($t_{\perp} = 0{,}3t$) dan kopling spin-orbit intrinsik ($\lambda_{\text{SO}} = 0{,}3\lambda_{\text{SO}}$). Simulasi dilakukan dengan metode \textit{tight-binding}. Proses \textit{averaging} dilakukan atas 500 realisasi disorder untuk meminimalisir fluktuasi statistik. \textbf{HASIL}. Hasil simulasi menunjukkan bahwa sistem mengalami perubahan struktur spektral pada $W > 6{,}0\lambda_{\text{SO}}$, ditandai dengan hilangnya persilangan pita tepi \textit{helical} akibat lokalisasi Anderson. Kemudian Peta spektrum dibangun dengan memvariasi $W$ dan $t_{\perp}$, mengungkapkan bahwa stabilitas QSHI tidak terpengaruhi seiring penambahan struktur antarlapisan. Analisis spektrum energi dan korelasi lokalisasi menunjukkan bahwa sistem mempertahankan keadaan tepi pada $W < 6{,}0\lambda_{\text{SO}}$, bahkan dengan peningkatan stabilitas hingga $t_{\perp} = 0{,}5t$. Temuan ini memberikan wawasan baru dalam desain material topologis berbasis struktur berlapis.

\vspace{0.2cm}
\noindent \textbf{Kata Kunci}: \textit{Quantum Spin Hall Insulator}, grafena, model Kane-Mele, disorder Anderson, simulasi \textit{tight-binding}
\end{abstract}

\newpage
\begin{abstract}
  \justifying AKMAL SURATMI. \textbf{Spectral Response of Quantum Spin Hall Stability on 3D Kane-Mele Graphene Model to Anderson Disorder}(supervised by Tasrief Surungan)
\par
\noindent
\textbf{INTRODUCTION}. This study investigates the stability of the \textit{Quantum Spin Hall Insulator (QSHI)} spectrum in a three-dimensional (3D) Kane-Mele graphene model with periodic stacking along the $z$-axis under Anderson localization.\textbf{AIM}. The primary objectives are to study disorder strength ($W$) effect in the 3D system and observe the degradation of \textit{helical edge states} in the energy band structure.\textbf{METHOD}. The 3D model is constructed by stacking two-dimensional (2D) graphene layers with interlayer hopping ($t_{\perp} = 0.3t$) and intrinsic spin-orbit coupling ($\lambda_{\text{SO}} = 0.3\lambda_{\text{SO}}$). Numerical simulations employ the \textit{tight-binding method}. Disorder averaging is performed over 500 configurations to minimize statistical fluctuations. \textbf{RESULT}Simulation results reveal a spectral structure change at $W > 6.0\lambda_{\text{SO}}$, characterized by the disappearance of helical edge due to Anderson localization. A spectral map constructed by varying $W$ and $t_{\perp}$ demonstrates enhanced QSHI stability with increased interlayer coupling. Energy spectrum analysis and localization correlations confirm the persistence of edge states for $W < 6.0\lambda_{\text{SO}}$, with stability further improved up to $t_{\perp} = 0.5t$. These findings provide critical insights for designing topological materials with layered architectures.

\vspace{0.2cm}
\noindent \textbf{Keywords}: Quantum Spin Hall insulator, graphene, Kane-Mele model, Anderson disorder, tight-binding simulation
\end{abstract}
