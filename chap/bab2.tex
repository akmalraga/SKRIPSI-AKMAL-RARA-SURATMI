\chapter{MODEL DAN METODE PENELITIAN}
\RaggedRight\section{Model Fisik Sistem}

\subsection{Tight Binding Model Grafena}
\justifying

Penelitian ini menggunakan pendekatan Tight Binding Model untuk merepresentasikan hamiltonian penuh dari material. Model ini dipilih, selain karena digunakan dalam rujukan inti \texttt{PhysRevB.91.085106} dan \parencite {PhysRevLett.95.226801}, namun juga karena sifat elektronik grafena yang hanya dipengaruhi oleh orbital $P_z$, sehingga material tersebut sangat efektif dibawah kedalam bentuk Tight Binding untuk mempermudah penyelesaian persamaan schrodinger.

Model Tight Binding merepresentasikan atom dalam kisi dengan satu basis orbital $|f_n\rangle = |\phi(x - a)\rangle$ yang terlokalisasi, dengan satu basis orbital setiap atom, orbital ini akan saling tumpang tindih dengan basis atom terdekat. Sehingga Hamiltonian sistem dapat dibangun dengan dua suku; suku lokal ditambah suku tumpang tindih \parencite{GROSSO20141}:
\[
  H = \sum_n \epsilon_n |f_n\rangle \langle f_n| + \sum_n b_{n+1} |f_n\rangle \langle f_{n+1}| + \text{H.c.}
\]
\par
Untuk Grafena, yang mana merupakan material dua dimensi dengan sruktur \textit{Honey comb}, orbital terluar atom akan tumpang tidih dengan tiga tetangga terdekat ditambah enam tetangga terdekat kedua.
\par
Dalam literatur modern \parencite{CastroNeto2009}, model hamiltonian Tight-Binding  sering dituliskan dalam bentuk kuantisasi kedua. Dimana dalam penggambaran mekanika kuantum, sistem ditentukan oleh fungsi gelombang, sedangkan dalam kuantisasi kedua, elektron dianihilasi dan diciptakan dalam medan fermion oleh operator $c$ dan $c^{\dagger}$. Sehingga secara matematis, Hamiltonian Grafena dapat diaproximasi dengan bentuk;
\[
\begin{aligned}
    H ={} & -t \sum_{\langle i,j \rangle,\sigma} \left( a^{\dagger}_{\sigma,i} b_{\sigma,j} + \text{H.c.} \right) 
    - t' \sum_{\langle\!\langle i,j \rangle\!\rangle,\sigma} \left( a^{\dagger}_{\sigma,i} a_{\sigma,j} + b^{\dagger}_{\sigma,i} b_{\sigma,j} + \text{H.c.} \right),
    \end{aligned}
\]

\subsection{Kane-Mele \textit{Spin Orbit Coupling}}
\justifying
Untuk memperoleh fase topologis, Kane-Mele\parencite{PhysRevLett.95.226801} memanfaatkan pengaruh \textit{Spin Orbit Coupling} pada grafena yang selama ini diabaikan sebab memiliki pengaruh yang sangat kecil pada sifat fisis grafena. Kane-Mele menggunakan dua replika model Haldane\citep {PhysRevLett.61.2015} yang menambahkan fase imajiner pada suku hopping tetangga kedua yang menciptakan medan magnet imajinatif pada ruang momentum, memodulasi gerakan elektron ketepi material. Kane-Mele meningkatkan model ini dengan menambahkan variabel \textit{SOC} dalam suku tersebut, menyebabkan modulasi berbeda setiap spin. secara matematis;

\[
  \begin{align}
  \mathcal{H}_{\text{KM},l} = t \sum_{\langle i,j \rangle,\sigma} c_{il\sigma}^\dagger c_{jl\sigma} 
+ i\lambda_{\text{SO}} \sum_{\langle\langle i,j \rangle\rangle,\sigma} \nu_{ij} c_{il\sigma}^\dagger s^z_{\sigma\sigma'} c_{jl\sigma'} \\
    + i\lambda_R \sum_{\langle i,j \rangle, \sigma \sigma'} c_{i\sigma}^\dagger (\mathbf{s} \times \hat{\mathbf{d}_{ij}})^z_{\sigma \sigma'}c_{j\sigma'}
+ \lambda_v \sum_{i,\sigma} \xi_i c_{il\sigma}^\dagger c_{il\sigma}.
\end{align}
\]

\par
Penambahan variabel spin di suku imajiner pada akhirnya akan mempertahankan simetri pembalikan waktu yang dirusak dalam model Haldane.

\[
\begin{align}
\mathcal{T} \, \mathcal{H}_{\mathrm{KM},l} \, \mathcal{T}^{-1}
&= \mathcal{T} \Bigg[
t \sum_{\langle i,j \rangle,\sigma} c_{il\sigma}^\dagger c_{jl\sigma}
+ i\lambda_{\mathrm{SO}} \sum_{\langle\langle i,j \rangle\rangle}
\nu_{ij} c_{il\sigma}^\dagger s^z_{\sigma\sigma'} c_{jl\sigma'} \nonumber \\
&\quad + i\lambda_R \sum_{\langle i,j \rangle}
c_{il\sigma}^\dagger (\mathbf{s} \times \hat{\mathbf{d}}_{ij})^z_{\sigma\sigma'} c_{jl\sigma'}
+ \lambda_v \sum_{i,\sigma} \xi_i c_{il\sigma}^\dagger c_{il\sigma}
\Bigg] \mathcal{T}^{-1} \\
&= t \sum_{\langle i,j \rangle,\sigma} c_{il\bar{\sigma}}^\dagger c_{jl\bar{\sigma}}
- i\lambda_{\mathrm{SO}} \sum_{\langle\langle i,j \rangle\rangle}
\nu_{ij} c_{il\bar{\sigma}}^\dagger s^z_{\bar{\sigma}\bar{\sigma}'} c_{jl\bar{\sigma}'} \nonumber \\
&\quad - i\lambda_R \sum_{\langle i,j \rangle}
c_{il\bar{\sigma}}^\dagger (\mathbf{s} \times \hat{\mathbf{d}}_{ij})^z_{\bar{\sigma}\bar{\sigma}'} c_{jl\bar{\sigma}'}
+ \lambda_v \sum_{i,\sigma} \xi_i c_{il\bar{\sigma}}^\dagger c_{il\bar{\sigma}} \\
&= \mathcal{H}_{\mathrm{KM},l}.
\end{align}
\]



\subsection{\textit{Stacking Layer}}
\label{subsec:modelstack}

\begin{figure}[h]
  \centering 
  \includegraphics[width=1.0\linewidth]{picture/stacking_layer.png}
  \caption{Visualisasi skema stacking layer}
  \label{fig:vis_stack}
\end{figure}

Kami membangun model tiga dimensi dengan menumpuk lapisan model Kane-Mele \parencite{PhysRevLett.95.226801}, mengikuti pendekatan peneleitian berikut \citep {PhysRevB.91.085106}. Hamiltonian total terdiri dari suku Kane-Mele intralayer, hopping antarlapisan, dan kopling spin-orbit (SOC) antarlapisan:

\[
\mathcal{H} = \underbrace{\sum_l \mathcal{H}_{\text{KM},l}}_{\text{Intralayer}} 
+ \underbrace{\tau \sum_{\langle l,l' \rangle} \sum_{i,\sigma} c_{il\sigma}^\dagger c_{il'\sigma}}_{\text{Hopping Antarlapisan}}
+ \underbrace{i\lambda_{SO\perp} \sum_{\langle l,l' \rangle} \sum_{i,\sigma,\sigma'} \mu_{ll'} c_{il\sigma}^\dagger s^z_{\sigma\sigma'} c_{il'\sigma'}}_{\text{Kopling Spin-Orbit Antarlapisan}},
\]

\par
Seperti yang ditunjukkan pada gambar \ref{fig:vis_stack}, \textit{stacking} dilakukan dengan munumpuk grafena dua dimensi di sumbu x, dengan perbedaan warna merah dan hijau menunjukkan perbedaan layer. skema ini menciptakan bentuk geometris tiga dimensi kepada grafena.

\section{Formulasi dan Validasi Numerik}
\subsection{Diskretisasi Ruang}
Dalam mekanika kuantum partikel diperlakukan sebagai variabel kontinyu, yang diatur oleh fungsi gelombang $\psi(x)$. Namun dalam kasus seperti material grafena, dimana basis orbital setiap atom terlokalisasi, ruang posisi dapat kita diskretisasi. Sehingga hamiltonian dari sistem juga dapat kita modelkan dalam bentuk matrix tridiagonal.
\[
H =
\begin{pmatrix}
\mathbf{H}_0 & \mathbf{V} & 0 & \cdots & 0 \\
\mathbf{V}^\dagger & \mathbf{H}_1 & \mathbf{V} & \cdots & 0 \\
0 & \mathbf{V}^\dagger & \mathbf{H}_2 & \cdots & 0 \\
\vdots & \vdots & \vdots & \ddots & \mathbf{V} \\
0 & 0 & 0 & \mathbf{V}^\dagger & \mathbf{H}_{N}
\end{pmatrix}.
\]

\par
Dengan model hamiltonian ini kita akan dengan mudah memperoleh nilai energi dan fungsi gelombang dari sistem, dengan memasukkan model ini kedalam komputer dan mendiagonalisasi hamiltonian dengan $H_n\psi_n(r)=E_n\psi(r)$.

\subsection{Validasi Efek Tepi}

Setelah memperoleh spektrum energi dari diagonalisasi hamiltonian pada ruang diskrit, kita akan memperoleh spektrum energi dari sistem yang periodik, dimana elektron memiliki potensi untuk menjelajah seluruh titik kisi dengan probabilitas universal. Namun dalam kasus konduktansi tepi, elektron tidak menjelajah seluruh titik kisi, melainkan termodulasi ketepian. Maka untuk memvalidasi model dan prosedur numerik, kita perlu menghitung spektrum energi sistem yang terbatas(tidak periodik). Hal ini dilakukan dengan \textit{slab geometry}, dimana ruang tiga dimensi direduksi menjadi dimensi yang lebih rendah.
\par
Untuk keperluan validasi, penelitian ini menggunakan arah (100), dimana arah y dan z tetap periodik sedangkan arah x terbatas. Secara konsep, seharusnya struktur pita akan menunjukkan terciptanya celah pita pada energi fermi akibat terlokalisasinya elektron di \textit{bulk}, namun akan ada dua pita yang berbelok dari valensi ke pita konduktif, \textit{vice verca}.

\par
Pada penelitian ini, efek dari \textit{Periodic Boundary Condition} dan \textit{Open Boundary Condition} juga diuji, untuk melihat bagaimana distribusi energi yang diperbolehkan pada sistem ketika diterapkan batas pada material dan bagaimana respon sistem ketika diberi kondisi periodik.


\section{Metode Analisis Spektrum}
Untuk melihat bagaimana respon sistem terhadap perubahan parameter, kita akan membangun peta respon energi. Hal ini dilakukan dengan membandingkan hasil plot fungsi energi terhadap variasi parameter. Pada penelitian ini, parameter yang akan divariasikan adalah stragged potential pada atom lokal dan kopling rashba, yang mana dijelaskan pada \parencite{PhysRevLett.95.226801}, melawan efek \textit{Spin Orbit Coupling}. Plot-plot energi akan disusun dalam sumbu stragged potential vs kopling rashba, untuk melihat di titik mana \textit{Spin Orbit Coupling} kalah.
\[
  \lambda_R < \lambda_{SOC}
\]
\subsection{Peta Spektrum Ribbon}
Peta pertama yang akan dibangun adalah spektrum ribbon. Seperti yang disinggung sebelumnya; \textit{edge state} hanya akan tervisualisasi pada spektrum pita energi apabila diterapkan \textit{slab geometry}. Untuk satu nilai parameter, plot akan dibangun dengan mendiagonalisasikan energi pada dua sumbu periodik dan satu sumbu terbatas di titik-titik simetri tinggi; ($\bar{\Gamma}$, $\bar{X}$, $\bar{M}$, $\bar{\Gamma}$,$\bar{Y}$)


\par 
Untuk menentukan kriteria dari \textit{edge state}, kita hanya perlu memperhatikan transisi pita konduksi menuju pita valensi, dan sebaliknya. di titik stragged potential dan rashba besar seharusnya sturktur pita akan menuju ke bentuk insulator trivial.
\subsection{Peta Probabilitas Elektron}
Hasil dari diagonalisasi matrix hamiltonian tidak hanya memberikan kita informasi eigen energi, tapi memberikan kita informasi dari eigen vektor(eigen state) $\psi_n(k)$. Eigen state($\psi_n(k)$) ini dapat ditulis sebagai superposisi orbital-orbital basis;
\[
|\psi_n\rangle = \sum_{\alpha}c_{n\alpha}|\phi_\alpha\rangle\]
\par
Nilai kuadrat dari $\sum_{\alpha}c_{n\alpha}$ secara fisis menunjukkan probabilitas keberadaan elektron, nilainya akan berkisar dari 0-1. Penelitian ini memanfaatkan nilai probabilitas ini untuk mendeteksi distribusi probabilitas pada sistem. Pita energi akan kembali diekstraksi dalam prosedur ini, namun dengan mode periodik dan penambahan pengukur probabilitas untuk melihat aktivitas elektron pada sistem.
\par
Untuk menentukan kriteria dari \textit{edge state} kita hanya perlu melihat pencampuran probabilitas pada struktur pita. Pada insulator trivial, probabilitas elektron akan kontras antara valensi dan konduksi, berbeda dengan mode tepi yang akan mencampur probabilitas.
\subsection{Peta Kerapatan Energi}
Pita energi hanya memberikan kita gambaran visual dari perilaku sistem, maka untuk memvalidasi keberadaan keadaan tepi(\textit{edge state}) kita perlu memperoleh nilai kuantitatif dari energi pada celah pita. Penelitian ini memanfaatkan ekstraksi kerapatan keadaan untuk mendeteksi keberadaan \textit{edge state} pada energi fermi. Lebih dari itu, tidak seperti pita energi yang memaksa kita mereduksi dimensi untuk keperluan visualisasi, kerapatan keadaan mengizinkan kita untuk memotong seluruh sumbu(111).
\par
Rapat keadaan dalam mekanika kuantum dimodelkan sebagai;
\[
  D(E) = \sum_n \delta (E - E_n)
\]
\par
Karena komputer tidak bisa membaca fungsi delta, maka penilitian ini menggunakan metode histogram untuk mendekati nilai analitik. Rentang energi akan dibagi menjadi bin dengan lebar $\Delta E$, hasil diagonalisasi hamiltonian kemudian akan menentukan berapa banyak level energi $E_n$ yang masuk kedalam setiap bin, yang kemudian akan dibagi dengan $\Delta E$ untuk memperoleh kerapatan energi.
\section{Metode Analisis Topologi}
\subsection{\textit{Hybrid Wannier Function}}
\label{subsec:HWF}
Penggunaan Hybrid Wannier Centers (HWC) dalam model stacking layer ini bertujuan untuk memetakan perubahan sifat topologi sistem saat dimensi material diperluas. Dengan mengamati pola winding pada spektrum Wilson Loop, perubahan fase topologis yang dipicu oleh interaksi antar-lapisan dapat diidentifikasi secara visual melalui transisi antara pola partner-switching (topologis) dan pola berpasangan (trivial), mengikuti prosedur \parencite{Cole_Python_Tight_Binding_2025}.
\par
Kane dan Mele \parencite{PhysRevLett.95.146802} menjelaskan bahwasanya invariant yang menjelaskan sifat topologi dari sistem \textit{Quantum Spin Hall} dapat ditentukan dari ganjil genap dari fluks, sehingga terdapat dua mode dari sistem ini; fase odd(Topologi, $Z_2=1$) dan fase even(Trivial Insulator, $Z_2=0$). Untuk memperoleh nilai dari $Z_2$, membagi ruang momentum menjadi dua sub-ruang; ruang titik TRIM(\textit{Time Reversal Symmetry}) dan sub-ruang dimana tumpang tindih antar matrix bersifat anti-unitarian($\Theta$). Untuk matrix anti-simetrik $2 \times 2$, seluruh informasi dapat diperoleh dari Pfaffian;
\[
P(\mathbf{k}) = Pf[\langle u_i(\mathbf{k})|\bar{\theta}|u_j(\mathbf{k})\rangle])
\]

Lewat perolehan ini, $Z_2$ invariant akhirnya dapat diperoleh dengan melihat perpindahan Pfaffian pada titik TRIM menghasilkan nol yang genap, maka sistem disebut even mode, begitupun sebaliknya.

\[
  \Delta = \frac{1}{2i\pi}\ointop_{\partial\tau} d\mathbf{k}.\nabla_{\mathbf{k}} \log [P(\mathbf + i\delta)] \hspace{1 cm} \text{mod 2},
  \]
\par
Meskipun formulasi Pfaffian memberikan definisi matematis yang kokoh, implementasi numeriknya sering kali terkendala oleh ambiguitas fase fungsi Bloch. \parencite{PhysRevB.83.035108} menawarkan pendekatan alternatif melalui representasi Wannier. Dalam metode ini, invarian $Z_2$ ditentukan melalui evolusi \textit{Wannier Charge Centers} (WCC) dalam skema \textit{hybrid}. Evolusi posisi pusat muatan ini terhadap momentum k pada arah tegak lurusnya menunjukkan 'pelintiran' topologis; di mana pada fase topologis ($Z_2$=1), jalur-jalur WCC akan bertukar pasangan Kramers dan menghasilkan jumlah persilangan yang ganjil pada garis referensi di Zona Brillouin.
\par
Dalam ruang satu dimensi, WCC dapat didefinisikan sebagai suku integrasi dari Potensial Berry\parencite{10.1098/rspa.1984.0023}\parencite{PhysRevB.95.075146};
\[
  \bar{x}_n = \frac{ia_x}{2\pi} int_{-\pi / a_x}^{\pi / a_x} dk_x \mathcal{A}_n (k_x)
\]
Atau;
\[
  \bar{x}_n(k_y, k_z) = \frac{a_x}{2\pi} \int_{-\pi / a_x}^{\pi / a_x} dk_x \mathcal{A}_n (k_x, k_y, k_z)

  \]
Dimana $\mathcal{A}_n(k)$ dituliskan sebagai; \[ \bar{A}_n (\bar{k}) =  \langle n_k | \nabla_{\bar{R}} | n_k \rangle \]
\par
Untuk membawa persamaan ini ke dalam perhitungan numerik, 
kita tidak dapat langsung mengevaluasi turunan 
$\nabla_{\bar{R}}$, sehingga koneksi Berry 
didefinisikan melalui transport paralel diskret antar titik 
$\mathbf{k}$ pada kisi Brillouin \parencite{doi:10.1143/JPSJ.74.1674}. 
Didefinisikan \textit{link variable}

\[
U_n(k_l) 
= \frac{\langle n(k_l) | n(k_l+\hat{\mu}) \rangle}
{|\langle n(k_l) | n(k_l+\hat{\mu}) \rangle|}
\]

yang merepresentasikan holonomi fase Berry sepanjang satu 
langkah mesh. Hubungan antara $U_n$ dan koneksi Berry kontinu 
diberikan oleh

\[
\langle n(k) | n(k+\delta k_\mu) \rangle 
\simeq \exp\!\left[-i A_\mu(k)\,\delta k_\mu \right],
\]

sehingga dengan mengikuti prosedur \parencite{Cole_Python_Tight_Binding_2025} koneksi Berry diskret dapat diekstraksi sebagai

\[
A_\mu(k) 
= -\frac{1}{i\,\delta k_\mu}\,\log U_n(k).
\]

Dalam implementasi numerik yang digunakan oleh paket \parencite{Cole_Python_Tight_Binding_2025}, ekstraksi nilai $A_{\mu}(k)$ melalui prosedur diskretisasi ini disusun menjadi sebuah operator transport paralel non-Abelian yang dikenal sebagai operator Wilson Loop, $D_{\perp}(k)$. Berdasarkan formulasi \parencite{PhysRevB.84.075119}, operator ini didefinisikan sebagai produk dari \textit{link variables} di sepanjang jalur tertutup dalam zona Brillouin:
D(k_\perp) = U(k_1) U(k_2) \dots U(k_N)
\]

Matriks Wilson Loop ini memiliki nilai eigen berbentuk $\lambda_m = e^{i\theta_m(k_\perp)}$, di mana fase $\theta_m$ merupakan fase Berry terakumulasi yang secara fisik merepresentasikan posisi pusat muatan Wannier hibrida (\textit{Hybrid Wannier Centers}) dalam satuan tanpa dimensi:

\[
\bar{x}_m(k_\perp) = \frac{\theta_m(k_\perp)}{2\pi} a_x
\]

Dengan demikian, prosedur Soluyanov-Vanderbilt untuk menentukan invarian $Z_2$ dilakukan dengan memplot evolusi fase $\theta_m$ terhadap momentum tegak lurusnya ($k_\perp$). Dalam fase topologis ($Z_2=1$), spektrum fase ini akan menunjukkan fenomena \textit{partner switching}, di mana jalur-jalur pusat muatan Wannier saling bertukar pasangan Kramers. Secara praktis, algoritma ini memastikan bahwa perhitungan tetap \textit{gauge-invariant}, sehingga sifat topologi material dapat diidentifikasi secara stabil tanpa memerlukan syarat \textit{gauge-fixing} yang rumit.

\subsection{\textit{Axion Angle}}
Gerakan orbital elektron dalam kristal tiga dimensi menciptakan \textit{Pseudoscalar magnetoelectric phase coupling} $\theta$ \parencite{PhysRevLett.102.146805}. Perhitungan $\theta$ adalah cara sederhana untuk meninjau adanya \textit{surface hall conductivity} pada material \parencite{PhysRevLett.58.1799}.
\[
  \delta \mathcal{L}_{EM} = \frac{\theta e^2}{2\pi h}\text{\textbf{E . B}}
\]
 Dengan itu \parencite{PhysRevB.78.195424} medefinisikan $\theta$ sebagai integrasi elektron dalam dimensi yang satu kali lebih tinggi, dengan koneksi Berry $\mathcal{A}^{\mu\nu}_j = i \langle u_{\mu} | \partial_j | u_{\nu} \rangle$, diperoleh
\[
  \theta = \frac{1}{2\pi} \int_{BZ} d^3 k \epsilon_{ijk} \text{Tr}[\mathcal{A}_i \partial_j \mathcal{A}_k - i \frac{2}{3}\mathcal{A}_i\mathcal{A}_j\mathcal{A}_k]
\]

\parencite{PhysRevB.78.195424} menunjukkan bahwa sistem topologis yang invarian terhadap pembalikan waktu berada pada dimensi 4 + 1 yang diklasifikasikan dengan \textit{second chern number}. Dengan itu $\theta(\beta)$ dihitung untuk ekspresi \textit{second chern number} menggunakan \textit{Four Curvature Formulation}
\[
  \theta = \frac{1}{16\pi} \int_{0}^{\beta} d\beta' \int d^3k \epsilon_{ijkl} \text{Tr}[\mathcal{F}_{ij}(\mathbf{k}, \beta')\mathcal{F}_{kl}(\mathbf{k}, \beta')]

\]

\section{Parameter dan Implementasi}
\subsection{Tabel Parameter}
\begin{table}[htbp]
\centering
\caption{Parameter Hamiltonian sistem \textit{stacking layer} Honeycomb.}
\label{tab:param_hamiltonian}
\begin{tabular}{@{}lccc@{}}
\toprule
\textbf{Parameter} & \textbf{Simbol} & \textbf{Nilai} & \textbf{Satuan/Keterangan} \\ \midrule
\textit{Hopping} dasar & $t$ & 1.0 & Unit energi ($e.V$) \\
\textit{Spin-Orbit Coupling} & $\lambda_{SOC}$ & 0.25 -- 0.30 & Nilai paper Kane-Mele \\
\textit{Rashba Coupling} & $\lambda_{R}$ & 0.0 -- 0.30 & Pecahnya simetri inversi \\
\textit{Staggered Potential} & $\Delta$ & 0.0 -- 3.0 & Perbedaan energi subkisi \\

Amplitudo Modulasi & $m_{axion}$ & 0.5 & Onsite potensial terhadap $\beta$ \\ \bottomrule
\end{tabular}
\end{table}

\begin{table}[htbp]
\centering
\caption{Konfigurasi geometri dan parameter kisi.}
\label{tab:param_geometri}
\begin{tabular}{@{}ll@{}}
\toprule
\textbf{Parameter} & \textbf{Deskripsi/Nilai} \\ \midrule
Struktur Kristal & Hexagonal (\textit{Honeycomb-like}) \\
Vektor Kisi Dasar & $\vec{a}_1 = (1, 0, 0)$, $\vec{a}_2 = (0.5, \sqrt{3}/2, 0)$, $\vec{a}_3 = (0, 0, 1)$ \\
Posisi Orbital & $A = (1/3, 1/3, 0)$, $B = (2/3, 2/3, 0)$ \\
Derajat Kebebasan & \textit{Spinful} (Matriks Pauli $\sigma_{x,y,z}$) \\
Arah Periodik & [0, 1, 2] (3D \textit{Bulk}) \\
Dimensi Geometri \textit{Slab} & 20 unit sel (Arah-\textit{y} terbatas) \\ \bottomrule
\end{tabular}
\end{table}

\begin{table}[htbp]
\centering
\caption{Parameter numerik dan konfigurasi simulasi \texttt{PythTB}.}
\label{tab:param_numerik}
\begin{tabular}{@{}lll@{}}
\toprule
\textbf{Parameter} & \textbf{Variabel Kode} & \textbf{Nilai/Pengaturan} \\ \midrule
Grid Momentum (HWC) & \texttt{mesh shape} & $41 \times 41 \times 41$ \\
Grid Momentum (Axion) & \texttt{nks} & $30 \times 30 \times 30$ \\
Rentang Parameter $\beta$ & \texttt{betas} & $[0, 2\pi]$ \\
Jumlah Sampling $\beta$ & \texttt{n\_beta} & 21 titik \\
Skema Diferensiasi & \texttt{diff\_scheme} & \textit{Central} \\
Orde Diferensiasi & \texttt{diff\_order} & 8 \\
Arah Integrasi WCC & \texttt{axis\_idx} & 1 (Sumbu-$y$) \\ \bottomrule
\end{tabular}
\end{table}

\subsection{Ekstensi Numerik}
Dalam mengimplementasikan model yang telah dibangun pada \ref{subsec:modelstack}, penelitian ini memanfaatkan paket PythTB 2.0 Python yang dikembangkan oleh grub peneliti \parencite{Cole_Python_Tight_Binding_2025}. Paket ini dibangun untuk mengonstruksi dan menganalisis model Tight-Binding, yang di khususkan untuk studi teoritikal material topologis. Paket ini mereduksi kode numerik yang panjang menjadi baris singkat, menghemat waktu dan tenaga peneliti.
\par
Penelitian ini mengadapsi empat fungsionalitas dari paket PythTB; Konstruksi model, \textit{State sampling}, Analisis topologi, dan Visualisasi. Pada tabel berikut \ref{tab:kodex}, kami menunjukkan struktur dan fungsionalitas dari kode yang kami kembangkan.
\begin{table}[htbp]
\centering
\caption{Struktur kode numerik \texttt{PythTB}.}
\label{tab:kodex}
\begin{tabularx}{\textwidth}{@{}l X X @{}} 
\toprule
\textbf{Kode} & \textbf{Fungsi} & \textbf{Keluaran} \\ \midrule
\texttt{Model-Hamiltonian.py} & Mengonstruksi model sistem \textit{tight-binding} & Fungsi objek model \\ \addlinespace
\texttt{Peta-spektrum.py} & Memetakan pengaruh gangguan (\textit{disorder}) terhadap efek tepi & Visualisasi peta spektrum energi \\ \addlinespace
\texttt{Peta-prob.py} & Memetakan pengaruh gangguan terhadap probabilitas elektron pada pita \textit{bulk} & Visualisasi distribusi probabilitas elektron \\ \addlinespace
\texttt{Dos.py} & Memetakan pengaruh gangguan terhadap kerapatan energi (\textit{Density of States}) & Peta \textit{Density of States} (DOS) \\ \addlinespace
\texttt{Hwf.py} & Menganalisis sifat topologi tiga dimensi melalui \textit{Wannier functions} & Plot evolusi \textit{Hybrid Wannier Centers} (HWC) \\ \addlinespace
\texttt{Axion-angle.py} & Menganalisis efek topologi melalui respon magnetoelektrik & Plot nilai $\theta$ terhadap parameter $\beta$ \\ \addlinespace
\texttt{Visualisasi.py} & Melakukan pemetaan struktur geometri model & Representasi visual struktur kristal \\ \bottomrule
\end{tabularx}
\end{table}

\section{Diagram Alur Penelitian}

\begin{figure}[htbp]
\centering
\begin{tikzpicture}[node distance=1.8cm, auto,
    % Definisi Style Blok
    startstop/.style={rectangle, rounded corners, minimum width=3cm, minimum height=1cm, text centered, draw=black, fill=red!10},
    process/.style={rectangle, minimum width=4cm, minimum height=1cm, text centered, draw=black, fill=blue!5},
    decision/.style={diamond, minimum width=3cm, minimum height=1cm, text centered, draw=black, fill=green!10, aspect=2},
    arrow/.style={thick,->,>=stealth},
    line/.style={thick,-}
]

% Node - Node
\node (start) [startstop] {Mulai};

\node (lit) [process, below of=start] {Studi Literatur \& Parameterisasi ($t, \lambda_{SO}, \lambda_R$)};

\node (model2d) [process, below of=lit] {Konstruksi Model 2D (Kane-Mele)};

\node (stack) [process, below of=model2d] {Stacking Layer (3D Model)};

\node (val) [decision, below of=stack, yshift=-0.5cm] {Validasi Efek Tepi?};

\node (analisis) [process, below of=val, yshift=-0.5cm] {Analisis Respon Sistem};

% Percabangan Analisis (Parallel)
\node (spektrum) [process, below of=analisis, xshift=-3cm, text width=3.5cm] {Peta Spektrum \& Probabilitas\\(Deteksi \textit{Edge State})};
\node (topologi) [process, below of=analisis, xshift=3cm, text width=3.5cm] {Analisis Topologi\\(HWC \& Axion $\theta$)};

\node (kesimpulan) [process, below of=analisis, yshift=-3cm] {Interpretasi Fisik \& Kesimpulan};

\node (stop) [startstop, below of=kesimpulan] {Selesai};

% Garis Penghubung
\draw [arrow] (start) -- (lit);
\draw [arrow] (lit) -- (model2d);
\draw [arrow] (model2d) -- (stack);
\draw [arrow] (stack) -- (val);

% Logika Decision
\draw [arrow] (val) -- node[anchor=east] {Ya} (analisis);
\draw [line] (val.east) -- ++(1.5,0) node[anchor=south] {Tidak};
\draw [arrow] ($(val.east)+(1.5,0)$) |- (model2d.east);

% Percabangan
\draw [arrow] (analisis) -| (spektrum);
\draw [arrow] (analisis) -| (topologi);

% Penyatuan Kembali
\draw [arrow] (spektrum) |- (kesimpulan);
\draw [arrow] (topologi) |- (kesimpulan);
\draw [arrow] (kesimpulan) -- (stop);

\end{tikzpicture}
\caption{Diagram alur tahapan penelitian dari konstruksi model hingga analisis topologi.}
\label{fig:alur_penelitian}
\end{figure}

\begin{figure}[htbp]
\centering
\begin{tikzpicture}[node distance=1.5cm,
    % Definisi Style
    io/.style={trapezium, trapezium left angle=70, trapezium right angle=110, minimum width=3cm, minimum height=1cm, text centered, draw=black, fill=orange!10},
    process/.style={rectangle, minimum width=3cm, minimum height=1cm, text centered, draw=black, fill=cyan!5},
    subroutine/.style={rectangle, rounded corners, minimum width=3cm, minimum height=1cm, text centered, draw=black, dashed},
    arrow/.style={thick,->,>=stealth}
]

% Input
\node (param) [io] {Input Parameter ($H_{intralayer}, H_{interlayer}$)};

% Core Process
\node (pythtb) [process, below=0.8cm of param] {\textbf{PythTB Engine}\\(Konstruksi Hamiltonian)};

% Output Intermediate
\node (diag) [process, below=0.8cm of pythtb] {Diagonalisasi Matriks ($H\psi = E\psi$)};

% Cabang Output
\node (dos) [subroutine, below left=1.5cm and -1cm of diag, text width=3cm] {\texttt{Dos.py}\\Hitung Kerapatan Keadaan};
\node (band) [subroutine, below=1.5cm of diag, text width=3cm] {\texttt{Peta-spektrum.py}\\Plot Pita Energi};
\node (hwf) [subroutine, below right=1.5cm and -1cm of diag, text width=3cm] {\texttt{Hwf.py}\\Hitung Pusat Wannier (HWC)};

% Final Result
\node (visual) [io, below=4.5cm of diag, minimum width=8cm] {Visualisasi Data \& Grafik ($\theta$, $Z_2$, Bandgap)};

% Garis
\draw [arrow] (param) -- (pythtb);
\draw [arrow] (pythtb) -- (diag);
\draw [arrow] (diag) -| (dos);
\draw [arrow] (diag) -- (band);
\draw [arrow] (diag) -| (hwf);

\draw [arrow] (dos) -- (visual.north -| dos);
\draw [arrow] (band) -- (visual.north -| band);
\draw [arrow] (hwf) -- (visual.north -| hwf);

\end{tikzpicture}
\caption{Skema komputasi numerik menggunakan paket PythTB untuk ekstraksi properti topologi.}
\label{fig:alur_numerik}
\end{figure}
