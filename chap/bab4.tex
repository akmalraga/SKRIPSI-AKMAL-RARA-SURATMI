\chapter{KESIMPULAN}
\RaggedRight\section{Kesimpulan}
\justifying
Setelah melakukan penelitian ini kami melihat adanya transisi dari struktur spektrum pada sistem Kane-Mele \textit{Stacking} pada titik kekuatan disorder $6\lambda_{SO}$ setelah titik itu sistem akan bertransisi dari struktur spektrum \textit{Quantum Spin Hall} menuju struktur trivial, hal ini menunjukkan bahwa efek kopling antar layer tidak mempengaruhi kestabilan pada sistem perlayer.
\par
Setelah menganalisa struktur pita pada sistem pada disorder kritis, sistem menunjukkan hilangnya \textit{Helical Edge States} pada struktur \textit{Ribbon}, hal ini memperkuat klaim kita bahwa nilai disorder $6\lambda_{SO}$ adalah titik transisi yang memisahkan struktur spektrum topologis dan fase trivial. Hasil dari struktur \textit{Ribbon} pita ini menunjukkan bahwa sistem dengan \textit{slab geometry} (010) terlokalisasi yang artinya sistem tidak memasuki fase metalik. Sehingga meskipun analisis Dos menunjukkan kenaikan \textit{Density Of States} pada titik kritis, namun sistem tetap terlokalisasi.
\par
Setelah Menyusun Pemetaan spektrum untuk menunjukkan menunjukkan stabilitas \textit{Quantum Spin Hall} pada model Kane-Mele \textit{Stacking} dengan melakukan \textit{Quench} pada kopling sistem, kami menyimpulkan bahwa sistem ini sangatlah stabil pada daerah disorder yang rendah yang pada aplikasinya merupakan efek dari kecacatan kisi. Hal ini menunjukkan bahwa model ini mampu menunjang \textit{Dissipationless electronics} di masa depan. 
\section{Saran}
Kami Sangatlah mengharapkan kelanjutan dari penelitian ini, agar lebih siap menuju aplikasi industrial. Salah satu \textit{Gap} utama dari penelitian ini adalah tidak adanya analisis dari konduktansi yang sangat vital pada topik ini, alasan dari tidak dilakukannya analisis ini adalah keterbatasan waktu dan kompleksitas dari analisis tersebut.
