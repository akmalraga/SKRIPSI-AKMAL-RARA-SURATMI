\pagenumbering{arabic}
\setcounter{page}{1}
\chapter{PENDAHULUAN}

\RaggedRight\section{Latar Belakang}
\justifying
Material Kuantum datang sebagai pahlawan ketika komponen elektronik tradisional mencapai limit miniaturisasi. Di tengah maraknya riset tentang quantum transistors yang memanfaatkan fenomena superkonduktivitas, entanglement, dan fase topologis\parencite{Chang2015}, grafena mencuat sebagai primus inter pares: material dua dimensi yang tidak hanya menjanjikan transisi dari elektronik ke spintronik, tetapi juga membuka pintu bagi realisasi dissipationless electronics melalui proteksi topologi  \parencite{PhysRevLett.95.226801}. Di sinilah fisika material bertemu dengan keanggunan matematika topologi, menciptakan simfoni sains yang berpotensi merevolusi komputasi kuantum hingga teknologi energi bersih\parencite{Nadeem2021}.
\par
Grafena menunjukkan reputasi yang cemerlang sebagai laboratorium para peneliti sejak beberapa puluh tahun belakangan, hal ini disebabkan oleh sifat fisis grafena yang memukau. Dari segi ketahanan grafena bahkan mengalahakan berlian yang juga tersusun dari jenis atom yang sama. lebih dari itu, grafena juga menunjukkan konduktifitas yang tinggi yang ditunjukkan pada struktur pita energi grafena yang menunjukkan Dirac cone pada titik simetri BZ \parencite{PhysRev.71.622}\parencite{PhysRevB.1.4747}. Reputasi konduktivitas grafena semakin melejit pada akhir abad ke 20 saat Haldane menunjukkan apa yang dia sebut parity anomaly, model yang dia tunjukkan adalah kisi honeycomb yang ada pada Graphite 2 dimensi(Grafena) dengan dua atom per sub-kisi yang di diatur untuk sedimikian untuk mendukung terealisasinya proteksi topologi, hasilnya menunjukkan bahwa dispersi pita terjadi pada Dirac cone dan menciptakan fenomena konduktansi hall yang terkuantisasi \parencite{PhysRevLett.61.2015}. Fenomena ini menarik kita kembali pada Quantum Hall Conductance yang diteliti sebelum Haldane.
\par
    The Beauty of Physics adalah kalimat yang disandingkan dengan fenomena Quantum Hall Conductance oleh banyak ahli, hal ini didasarkan pada kuantisasi yang terjadi pada konduktansi ($\sigma_{xy}=\nu \frac{e^2}{\hbar}$) yang artinya pada regim kuantum detail material tidaklah menjadi penentu sifat elektronik melainkan hanya ditentukan oleh sifat topologis sistemnya\parencite{PhysRevLett.45.494}. Thouless pada penelitiannya kemudian menjelaskan fase Topologis, dimana sistem dibedakan dengan TKKN(Thouless-Kohmoto-Nightingale-den Nijs) sebuah integer yang diperoleh dengan mengintegralkan medan magnetik pada BZ, Hal ini didasarkan pada perhitungan genus yang membedakan sifat topologis sistem, inilah kemudian melahirkan istilah Topological Insulators \parencite{PhysRevLett.49.405}. 
    \par
    Topik ini terus berkembang hingga pada tahun 2005 Kane dan Mele membuka pintu baru pada penelitian ini, Kopling Spin-Orbit yang selama ini diabaikan pada grafena sebab kecilnya efeknya tidak memberikan kontribusi yang signifikan pada sifat fisis grafena. Namun, Kane dan Mele menunjukan peningkatan nilai kopling spin-orbit yang terjadi pada energi menuju nol mutlak menciptakan dispersi pita pada sistem persis dengan yang terjadi pada model Haldane. Fase Topologis yang ditunjukkan pada model Kane dan Mele menunjukkan ketahanan yang luar biasa terhadap gangguan fisis, hal ini dikarenakan fenomena spin split yang menjaga Topological invariant dari system \parencite{PhysRevLett.95.226801}\parencite{PhysRevLett.95.146802}\parencite{PhysRevLett.98.106803}.
\par 

Beberapa tahun setelah Kane dan Mele memperkenalkan $Z2$ invariant yang menunjukkan fenomena \textit{Quantum Spin Hall} dalam material dua dimensi, mereka melanjutkan penelitian mereka dalam ekstensi tiga dimensi, dalam paper \parencite{PhysRevLett.98.106803}. Namun ketertarikan kami pada penelitian ini terpusat pada ekstensi tiga dimensi yang dimodelkan pada penelitian \parencite{PhysRevB.91.085106} dengan menciptakan tumpukan lapisan grafena dua dimensi, hasilnya menunjukkan adanya efek topologis yang lemah, dengan kata lain tidak ada \textit{surface state} seperti yang ditunjukkan pada penelitian \parencite{PhysRevLett.98.106803}. Namun rangkaian penelitian yang dilakukan oleh \parencite{Vanderbilt2018} \parencite{10.1098/rspa.1984.0023} \parencite{PhysRevB.75.121403} \parencite{PhysRevB.83.035108} \parencite{PhysRevB.84.075119} \parencite{PhysRevLett.102.146805}, membuat kami ingin mempelajari lebih jauh respon parameter-parameter yang berkaitan dengan fase topologi grafena ketika terjadi tumpukan lapisan dua dimensi. Penelitian ini juga terinspirasi dari modul yang dikembangkan oleh \parencite{Cole_Python_Tight_Binding_2025} yang digunakan dalam studi fase topologis material.


\section{Landasan Teori}
\subsection{Sifat Elektronik Material}
\justifying 

Sifat elektronik material masih terhalang kabut yang tebal sebelum mekanika kuantum datang. Sebelumnya, sifat elektronik dari suatu material masih ditentukan oleh atom penyusun material tersebut. Paul Drude(1900) menjelaskan fenomena kelistrikan material dengan menganggap elektron sebagai gas gas ideal klasik, teori ini sayangnya gagal menjelaskan kapasitas panas elektronik \parencite{kittel1955solid}.

\par

Setelah datangnya mekanika kuantum kita tahu bahwa energi terkuantisasi pada level-level tertentu. Lebih dari itu, lewat representasi Bloch kita mampu membangun struktur pita energi dalam sistem material, hal ini akhirnya memberikan kita informasi terbaru perihal sifat elektronik material.

\par

Dalam menyelesaikan persamaan schrodinger, Bloch menggunakan fungsi gelombang periodik, dengan berlandaskan potensial periodik pada kristal.
\[
  \psi(r) = u_k(r) e^{ik.r}
\]
Interaksi antara gelombang elektron inilah yang akhirnya melahirkan celah pita energi. Klasifikasi material lahir dari informasi ini; isolator adalah material celah pita cukup besar antara pita valensi dan pita konduksi sehingga elektron tidak mampu berpindah bebas antara kisi, sedangkan konduktor adalah material dengan pita yang terisi setengah, sehingga elektron bebas bergerak di dalam material bahkan dengan energi tambahan yang kecil. Dari teori celah pita ini juga lahir semikonduktor, sebuah material dengan celah pita kecil sehingga elektron masih mampu melompat ke kisi tetangga.

\subsection{Efek Tepi dan Kuantisasi Konduktansi Tepi}
Ide dari konduktansi tepi telah lahir jauh sebelum material topologis datang. Pada fisika klasik, ketika sebuah medan listrik $E$ mengalir lurus kedalam sebuah pelat metal tipis sedemikan hingga elektron hanya mampu bergerak bebas dalam bidang dua dimensi. Kemudian sebuah medan magnet diaplikasikan tegak lurus terhadap pelat metal, gaya Lorentz akan memodulasi elektron menuju ketepian pelat\parencite{kittel1955solid}.
\[
  \begin{align}
    E_y = -\frac{eB\tau}{m}E_x \\ 
    R_H = \frac{E_y}{j_x B} \\ 
    R_H = - \frac{1}{ne}
    \rho_{xy} = R_H B_z
  \end{align}
\]
Deskripsi matematis ini memberi kita informasi dari resisvitas yang ditentukan secara linear oleh penambahan kekuatan medan magnet $B_z$.
\par 
\parencite{PhysRevLett.45.494} Kemudian mencoba membawa fenomena ini kedalam rezim kuantum. dan menunjukkan bahwa pada medan magnet kuat dan suhu rendah, koefisien Hall menjadi terkuantisasi. Kuantisasi dari gerakan orbital elektron dengan frekuensi siklotron $\omega_c$ mengarah ke kuantisasi level Landau dengan energi $\epsilon_m = \hbar \omega_c(m + 1/2)$. Fenomena ini melahirkan kuantisasi konduktansi tepi;
\[
  \sigma_{xy} = N e^2 / h
\]
\subsection{TKKN Invariant}
\label{subsec:TKKN}
Ketika visualisasi dilakukan terhadap pita energi pada sistem yang dikerjakan oleh \parencite{PhysRevLett.45.494}, akan diperoleh celah pita yang memperantarai pita energi. Pertanyaan yang muncul setelah penelitian penting ini adalah; apa yang membedakan sistem ini dengan insulator biasa dan darimana nilai kuantisasi ini diperoleh.
\par 

Jawaban dari teka-teki ini muncul lewat diksi matematis yang menyelinap masuk ke skena fisika. Topologi, sebagaimana dijelaskan oleh \parencite{PhysRevLett.49.405}, adalah diksi yang menjelaskan klasifikasi geometri yang tahan terhadap deformasi ringan, seperti benda dengan satu lubang akan tetap diklasifikasikan dengan "genus" satu, selama lubang tersebut tetap satu tidak peduli dengan bentuk kulitnya. Hal ini membantu kita menjelaskan bahwa selama dua pita memiliki properti topologi yang sama, keduanya mampu terkoneksi secara adiabatik tanpa penutupan celah.
\par 

Bilangan kuantum topologi yang mengikat pita-pita ini dapat dimengerti lewat formulasi Berry Phase\parencite{10.1098/rspa.1984.0023}. Berry menunjukkan pada penelitiannya bahwa sebuah hamiltonian kuantum akan menghasilkan fase geometris kompleks ketika diberi variasi eksternal. Fase ini memiliki garis integral $\mathcal{A}_m = i \langle u_m | \nabla_k | u_m \rangle$, dan integral bidang $\mathcal{F}_m = \nabla \times \mathcal{A}_m$. Invariant atau bilangan kuantum dari sistem dapat dihitung dengan;
\[
  n_m = \frac{1}{2\pi} \int d^2 \mathbf{k} \mathcal{F}
_m\]
\subsection{Model Haldane}
Salah satu contoh material yang secara struktur mengizinkan fenomena topologis seperti yang dijelaskan pada bagian \ref{subsec:TKKN} datang dari material dua dimensi Grafena \parencite{CastroNeto2009}\parencite{Novoselov2005}. Material ini sangat menarik dari segi sifat elektronik dimana sturuktur pitanya menunjukkan apa yang disebut \textit{Dirac Cone} pada sudut Brilioun Zone. Hal ini menyebabkan elektron bergerak mendekati kecepatan cahaya ketika melewati grafena. Fenomena ini juga makin jelas apabila melihat hamiltonian material ini, sebuah hamiltonian dengan susunan menyurupai hamiltonian relativistik Dirac tanpa massa; $\mathcal{H}(\mathbf{q} = \hbar v_F \mathbf{q} . \bar{\sigma})$
\par 

Ide Kuantisasi tepi ini kemudian dibawa oleh \parencite{PhysRevLett.61.2015} kedalam material grafena. Mengingat bahwa degenerasi di tengah titik \textit{Dirac Cone} dilindungi oleh simetri pembalikan waktu dan simetri inversi, maka Haldane mencoba untuk merusak simetri inversi dengan mengaplikasikan beda potensial pada dua atom basis pada subkisi grafena(di masa itu masih disebut Graphite 2D, karena \parencite{Novoselov2005} baru berhasil mengekstraksi Grafena beberapa tahun kemudian) hal ini menciptakan hamiltonian dengan massa $m$ yang berperan dalam penciptaan celah pada \textit{Dirac Cone} sebesar $2|m|$.
\par 

Ketika prosedur perusakan simetri inversi di atas menghasilkan celah energi, yang mana mengantarkan material tersebut menjadi insulator. Namun ketika medan magnet dengan rerata nol diaplikasikan untuk menciptakan fase imajiner kepada sistem, Haldane menunjukkan bahwa sistem tersebut tidak lagi menjadi sebuah insulator biasa. Dengan menggunakan model material teoritis dengan stuktur heksagonal, Haldane berhasil menciptakan Kuantisasi tepi $\sigma_{xy}= e^2/h$.

\par
Hasil yang diperoleh oleh Haldane dapat dikaitkan dengan perolehan bilangan kuantum yang didiskusikan pada \ref{subsec:TKKN}. Seperti yang ditunjukkan oleh \parencite{10.1098/rspa.1984.0023}, Fase periodik menyebabkan terpelintirnya fungsi Bloch dalam ruang reciprocal. Pelintiran fungsi oleh Fase Berry inilah yang bertanggung jawab sebagai medan magnet imajintif pada ruang momentum, memodulasi elektron menuju ketepian kristal.

\subsection{Model Kane dan Mele}
Didasari oleh keberhasilan ekstraksi material dua dimensi grafena oleh \parencite{Novoselov2005} dan eksotisme model \parencite{PhysRevLett.61.2015}, Kane dan Mele dalam dua papernya \parencite{PhysRevLett.95.226801}\parencite{PhysRevLett.95.146802} memperkenalkan apa yang disebut $Z2$ invariant yang merupakan klasifikasi baru untuk material topologis yang begitu kuat berkat perlindungan simetri pembalikan waktu.
\par 
Pada penellitian pertamanya \parencite{PhysRevLett.95.226801}, mereka menunjukkan bagaimana aplikasi dari dua model \parencite{PhysRevLett.61.2015} dan doping kopling orbit spin pada grafena menciptakan modulasi yang unik, dimana pemisahan spin ini memaksa adanya transfer spin up dan down ketepian material, namun modulasi arah yang berbanding terbalik antar dua spin membuat total \textit{chern number} menjadi nol. Kane mele juga menunjukkan adanya peralihan pita valensi menuju konduksi dan sebaliknya(yang merupakan konsekuensi \textit{Krammer's pair}) pada struktur pita ribbon sistem. Untuk memperolehnya, Kane dan Mele merumuskan model hamiltonian Tight-Binding
\[
  \begin{align}
  \mathcal{H}_{\text{KM},l} = t \sum_{\langle i,j \rangle,\sigma} c_{il\sigma}^\dagger c_{jl\sigma} 
+ i\lambda_{\text{SO}} \sum_{\langle\langle i,j \rangle\rangle,\sigma} \nu_{ij} c_{il\sigma}^\dagger s^z_{\sigma\sigma'} c_{jl\sigma'} \\
    + i\lambda_R \sum_{\langle i,j \rangle, \sigma \sigma'} c_{i\sigma}^\dagger (\mathbf{s} \times \hat{\mathbf{d}_{ij}})^z_{\sigma \sigma'}c_{j\sigma'}
+ \lambda_v \sum_{i,\sigma} \xi_i c_{il\sigma}^\dagger c_{il\sigma}.
\end{align}
\]

Di akhir mereka juga menunjukkan bagaimana perilaku spintronik ini dapat mengevolusi perangkat elektronik modern seperti pengembangan transistor spintronik yang \textit{dissipationless}.
\par 
Pada penelitian keduanya \parencite{PhysRevLett.95.146802}, Kane dan Mele secara resmi memperkenalkan klasifikasi topologis yang mereka temukan pada penelitian sebelumnya. Dengan menggunakan argumen Laughlin, Kane dan Mele menjelaskan bagaimana invariant pembalikan waktu dipengaruhi hanya oleh genap dan ganjil flux yang diberikan pada sistem silinder. Dari argumen inilah Kane dan Mele meyakinkan perihal klasifikasi $Z2$ yang hanya bernilai 0 dan 1. Untuk menghitung properti tersebut, Kane dan Mele menggunakan Pfaffian untuk melihat pelintiran fungsi gelombang pada ruang momentum. Namun pada penelitian lebih lanjut \parencite{Vanderbilt2018} \parencite{PhysRevB.75.121403} \parencite{PhysRevB.83.035108}, nilai invariant $Z2$ kerap diselidiki menggunakan parameter yang lebih mudah dan presisi secara numerik.

\section{Tujuan Penelitian}
\begin{enumerate}
  \item  Memperoleh struktur pita ribbon pada model  \textit{stacking} grafena dengan \textit{hopping} antar lapisan yang menunjukkan perilaku efek tepi sistem.
  \item Memperoleh peta respon sistem  \textit{stacking} grafena terhadap perubahan parameter \textit{hopping} dan potensial kisi lewat analisis sturktur pita, kerapatan energi, dan probabilitas elektron.
  \item Memperoleh plot evolusi  \textit{ Hybrid Wannier Function Center} terhadap pergerakan ruang momentum searah dengan arah penumpukan lapisan.
  \item Memvalidasi keberadaan atau ketiadaan \textit{Surface State} pada permukaan grafena \textit{Stacking} dengan analisis perilaku magnetoelektrik atau \textit{Axion Angle}.
\end{enumerate}
\section{Manfaat Penelitian}
\begin{enumerate}
\item \textbf{Pengembangan Teori Transisi Dimensi:} Memberikan wawasan teoretis mengenai bagaimana sifat topologi dua dimensi (model Kane-Mele) berevolusi dan berinteraksi saat disusun menjadi sistem tiga dimensi (\textit{stacking}) melalui parameter \textit{hopping} antar-lapisan.
\item \textbf{Panduan Desain Material Spintronik:} Menyediakan data dasar bagi perancangan material masa depan melalui pemetaan respon spektral, sehingga memungkinkan identifikasi kondisi ideal untuk mempertahankan \textit{surface states} yang stabil.
\item \textbf{Karakterisasi Invarian Topologi:} Memperjelas hubungan antara parameter fisis deterministik (seperti potensial kisi dan \textit{hopping}) dengan nilai \textit{Axion Angle} (θ) sebagai indikator utama respon magnetoelektrik sistem.
\item \textbf{Penyediaan Peta Parameter Fisis:} Menghasilkan profil respon sistem yang dapat digunakan sebagai acuan dalam melakukan penalaan parameter (\textit{parameter tuning}) pada studi eksperimental material isolator topologis.
\end{enumerate}
