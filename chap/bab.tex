\pagenumbering{arabic}
\setcounter{page}{1}
\chapter{PENDAHULUAN}

\RaggedRight\section{Latar Belakang}
\justifying
\lipsum

\section{Landasan Teori}
\subsection{Sifat Elektronik Material}
\justifying 

Sifat elektronik material masih terhalang kabut yang tebal sebelum mekanika kuantum datang. Sebelumnya, sifat elektronik dari suatu material masih ditentukan oleh atom penyusun material tersebut. Paul Drude(1900) menjelaskan fenomena kelistrikan material dengan menganggap elektron sebagai gas gas ideal klasik, teori ini sayangnya gagal menjelaskan kapasitas panas elektronik \parencite{kittel1955solid}.

\par

Setelah datangnya mekanika kuantum kita tahu bahwa energi terkuantisasi pada level-level tertentu. Lebih dari itu, lewat representasi Bloch kita mampu membangun struktur pita energi dalam sistem material, hal ini akhirnya memberikan kita informasi terbaru perihal sifat elektronik material.

\par

Dalam menyelesaikan persamaan schrodinger, Bloch menggunakan fungsi gelombang periodik, dengan berlandaskan potensial periodik pada kristal.
\[
  \psi(r) = u_k(r) e^{ik.r}
\]
Interaksi antara gelombang elektron inilah yang akhirnya melahirkan celah pita energi. Klasifikasi material lahir dari informasi ini; isolator adalah material celah pita cukup besar antara pita valensi dan pita konduksi sehingga elektron tidak mampu berpindah bebas antara kisi, sedangkan konduktor adalah material dengan pita yang terisi setengah, sehingga elektron bebas bergerak di dalam material bahkan dengan energi tambahan yang kecil. Dari teori celah pita ini juga lahir semikonduktor, sebuah material dengan celah pita kecil sehingga elektron masih mampu melompat ke kisi tetangga.

\subsection{Efek Tepi dan Kuantisasi Konduktansi Tepi}
Ide dari konduktansi tepi telah lahir jauh sebelum material topologis datang. Pada fisika klasik, ketika sebuah medan listrik $E$ mengalir lurus kedalam sebuah pelat metal tipis sedemikan hingga elektron hanya mampu bergerak bebas dalam bidang dua dimensi. Kemudian sebuah medan magnet diaplikasikan tegak lurus terhadap pelat metal, gaya Lorentz akan memodulasi elektron menuju ketepian pelat\parencite{kittel1955solid}.
\[
  \begin{align}
    E_y = -\frac{eB\tau}{m}E_x \\ 
    R_H = \frac{E_y}{j_x B} \\ 
    R_H = - \frac{1}{ne}
    \rho_{xy} = R_H B_z
  \end{align}
\]
Deskripsi matematis ini memberi kita informasi dari resisvitas yang ditentukan secara linear oleh penambahan kekuatan medan magnet $B_z$.
\par 
\parencite{PhysRevLett.45.494} Kemudian mencoba membawa fenomena ini kedalam rezim kuantum. dan menunjukkan bahwa pada medan magnet kuat dan suhu rendah, koefisien Hall menjadi terkuantisasi. Kuantisasi dari gerakan orbital elektron dengan frekuensi siklotron $\omega_c$ mengarah ke kuantisasi level Landau dengan energi $\epsilon_m = \hbar \omega_c(m + 1/2)$. Fenomena ini melahirkan kuantisasi konduktansi tepi;
\[
  \sigma_{xy} = N e^2 / h
\]
\subsection{TKKN Invariant}
\label{subsec:TKKN}
Ketika visualisasi dilakukan terhadap pita energi pada sistem yang dikerjakan oleh \parencite{PhysRevLett.45.494}, akan diperoleh celah pita yang memperantarai pita energi. Pertanyaan yang muncul setelah penelitian penting ini adalah; apa yang membedakan sistem ini dengan insulator biasa dan darimana nilai kuantisasi ini diperoleh.
\par 

Jawaban dari teka-teki ini muncul lewat diksi matematis yang menyelinap masuk ke skena fisika. Topologi, sebagaimana dijelaskan oleh \parencite{PhysRevLett.49.405}, adalah diksi yang menjelaskan klasifikasi geometri yang tahan terhadap deformasi ringan, seperti benda dengan satu lubang akan tetap diklasifikasikan dengan "genus" satu, selama lubang tersebut tetap satu tidak peduli dengan bentuk kulitnya. Hal ini membantu kita menjelaskan bahwa selama dua pita memiliki properti topologi yang sama, keduanya mampu terkoneksi secara adiabatik tanpa penutupan celah.
\par 

Bilangan kuantum topologi yang mengikat pita-pita ini dapat dimengerti lewat formulasi Berry Phase\parencite{10.1098/rspa.1984.0023}. Berry menunjukkan pada penelitiannya bahwa sebuah hamiltonian kuantum akan menghasilkan fase geometris kompleks ketika diberi variasi eksternal. Fase ini memiliki garis integral $\mathcal{A}_m = i \langle u_m | \nabla_k | u_m \rangle$, dan integral bidang $\mathcal{F}_m = \nabla \times \mathcal{A}_m$. Invariant atau bilangan kuantum dari sistem dapat dihitung dengan;
\[
  n_m = \frac{1}{2\pi} \int d^2 \mathbf{k} \mathcal{F}
_m\]
\subsection{Model Haldane}
Salah satu contoh material yang secara struktur mengizinkan fenomena topologis seperti yang dijelaskan pada bagian \ref{subsec:TKKN} datang dari material dua dimensi Grafena \parencite{CastroNeto2009}\parencite{Novoselov2005}. Material ini sangat menarik dari segi sifat elektronik dimana sturuktur pitanya menunjukkan apa yang disebut \textit{Dirac Cone} pada sudut Brilioun Zone. Hal ini menyebabkan elektron bergerak mendekati kecepatan cahaya ketika melewati grafena. Fenomena ini juga makin jelas apabila melihat hamiltonian material ini, sebuah hamiltonian dengan susunan menyurupai hamiltonian relativistik Dirac tanpa massa; $\mathcal{H}(\mathbf{q} = \hbar v_F \mathbf{q} . \bar{\sigma})$
\par 

Ide Kuantisasi tepi ini kemudian dibawa oleh \parencite{PhysRevLett.61.2015} kedalam material grafena. Mengingat bahwa degenerasi di tengah titik \textit{Dirac Cone} dilindungi oleh simetri pembalikan waktu dan simetri inversi, maka Haldane mencoba untuk merusak simetri inversi dengan mengaplikasikan beda potensial pada dua atom basis pada subkisi grafena(di masa itu masih disebut Graphite 2D, karena \parencite{Novoselov2005} baru berhasil mengekstraksi Grafena beberapa tahun kemudian) hal ini menciptakan hamiltonian dengan massa $m$ yang berperan dalam penciptaan celah pada \textit{Dirac Cone} sebesar $2|m|$.
\par 

Ketika prosedur perusakan simetri inversi di atas menghasilkan celah energi, yang mana mengantarkan material tersebut menjadi insulator. Namun ketika medan magnet dengan rerata nol diaplikasikan untuk menciptakan fase imajiner kepada sistem, Haldane menunjukkan bahwa sistem tersebut tidak lagi menjadi sebuah insulator biasa. Dengan menggunakan model material teoritis dengan stuktur heksagonal, Haldane berhasil menciptakan Kuantisasi tepi $\sigma_{xy}= e^2/h$.

\par
Hasil yang diperoleh oleh Haldane dapat dikaitkan dengan perolehan bilangan kuantum yang didiskusikan pada \ref{subsec:TKKN}. Seperti yang ditunjukkan oleh \parencite{10.1098/rspa.1984.0023}, Fase periodik menyebabkan terpelintirnya fungsi Bloch dalam ruang reciprocal. Pelintiran fungsi oleh Fase Berry inilah yang bertanggung jawab sebagai medan magnet imajintif pada ruang momentum, memodulasi elektron menuju ketepian kristal.

\subsection{Model Kane dan Mele}
